\documentclass[12pt]{report}

\usepackage{listings}
\usepackage{color}

\definecolor{dkgreen}{rgb}{0,0.6,0}
\definecolor{gray}{rgb}{0.5,0.5,0.5}
\definecolor{mauve}{rgb}{0.58,0,0.82}

\lstset{frame=tb,
  language=Java,
  aboveskip=3mm,
  belowskip=3mm,
  showstringspaces=false,
  columns=flexible,
  basicstyle={\small\ttfamily},
  numbers=none,
  numberstyle=\tiny\color{gray},
  keywordstyle=\color{blue},
  commentstyle=\color{dkgreen},
  stringstyle=\color{mauve},
  breaklines=true,
  breakatwhitespace=true,
  tabsize=3
}

\begin{document}
\tableofcontents

\chapter{Lexical Conventions}

\section{Comments}
In Jtrix, \texttt{/*} opens a multi-line comment and \texttt{*/} closes the comment. \texttt{//} opens a single line comment that is ended by a newline. Comments are ignored and they do not nest. Comments cannot be contained in literals.

\begin{lstlisting}
// This is a single line comment.

/* This is a multi-line comment,
	so I can end it down here. */

int x = 13; //Comments can follow lines of code
\end{lstlisting}

\section{White Space}
\begin{lstlisting}
//white space is defined as follows:

' '	    //space character
'\t'		//horizontal tab character
'\n'	//newline character
\end{lstlisting}
White space is ignored, but can be utilized in cases of tokenization.
\begin{lstlisting}
//white space is used to differentiate between the single token >= and the two tokens > and =

a >= b;   //this is a valid statement which evaluates whether a is greater than or equal to b, whereas
a > = b;  //is not a valid statement
\end{lstlisting}
\section{Tokens}
Jtrix uses five different tokens:

 \textit{Identifiers}

 \textit{Keywords}

 \textit{Literals}

 \textit{Separators}

 \textit{Operator}
\subsection{Identifiers}
Jtrix Identifiers are any \textit{n}-length set of characters beginning with an English letter that is either uppercase (\texttt{A - Z}) or lowercase (\texttt{a - z}) that do not form a Jtrix Keyword. Identifiers may not start with integers (\texttt{0 - 9}) or with underscores (\texttt{\_}) but can include them in any following position.
\begin{lstlisting}
//valid identifiers
thing
thing1
thingTwo
ThingTwo
Thing_Two

/* Note: identifiers are case sensitive, so thingTwo and ThingTwo are treated distinctly different */
\end{lstlisting}

\subsection{Keywords}
The following ASCII character sequences are reserved identifiers that may only be used as Jtrix Keywords:
 \color{blue}\texttt{break}	 		
~	 \color{blue}\texttt{boolean} 	
~	 \color{blue}\texttt{char} 
~	 \color{blue}\texttt{col} 
~	 \color{blue}\texttt{continue} 
~	 \color{blue}\texttt{else} 
~	 \color{blue}\texttt{float} 
~		 \color{blue}\texttt{for} 
~	 \color{blue}\texttt{foreach}
~	 \color{blue}\texttt{if} 
~	 \color{blue}\texttt{int} 
~	 \color{blue}\texttt{matrix} 
~	 \color{blue}\texttt{return} 
~	 \color{blue}\texttt{row} 
~	 \color{blue}\texttt{string} 
~	 \color{blue}\texttt{void} 
~	 \color{blue}\texttt{while}


\subsection{Operators}
Jtrix has both binary and unary operators. Operators are used in combination with variables or literals to create expressions. Binary operators are represented and evaluated in infix notation and unary operators are evaluated in prefix notation. Jtrix uses the following types of operators:

 \textit{Arithmetic Operators}

 \textit{Equivalence Operators}

 \textit{Logical Operators}

\subsubsection{Arithmetic Operators}
\begin{lstlisting}
// addition applies to int, float, matrix
	+
// subtraction applies to int, float, matrix
	-
// multiplication applies to int, float, matrix
	*
// division applies to int, float
	/
// modulo applies to int
	%
\end{lstlisting}
\subsubsection{Equivalence Operators}
\begin{lstlisting}
// equality applies to int, boolean, float, matrix
	==
// inequality applies to int, boolean, float, matrix
	!=
// greater than applies to int, float
	>
// greater than or equal to applies to int, float
	>=
// less than applies to int, float
	<
// less than or equal to applies to int, float
	<=
\end{lstlisting}

\subsubsection{Logical Operators}
\begin{lstlisting}
// all logic operators apply only to boolean

// AND
	&&
// OR
	||
// NOT
	!
\end{lstlisting}

\subsubsection{Unary Operators}
The operator not ( \texttt{!} ) is used strictly as a unary operator. The minus sign ( \texttt{-} ) can be used as both a unary and binary operator.
\begin{lstlisting}
boolean a = true;
return !a; // using not as a unary operator

int pos1 = 1; 
int neg1 = -1;  // using the minus sign as a unary operator
return (neg1 - pos1); // using the minus as a binary operator
\end{lstlisting}

\subsubsection{Precedence of Operators}
\begin{lstlisting}
// Jtrix will evaluate operators in the following order:
 // 1. NOT, and arithmetic negation
!  - 

// 2. multiplication, division, and modulo
*  /  %

// 3. addition, and subtraction
+  - 

// 4. equal to, and not equal to
==  !=

// 5. less than, less than or equal to, greater than, and greater than or equal to
<  <=  >  >=

// 6. AND
&&

// 7. OR
||
\end{lstlisting}

\subsection{Literals}
Jtrix uses the following literals:

 \textit{Integer}

 \textit{Float}

 \textit{Boolean}

 \textit{Character}

 \textit{String}

\subsubsection{Integer Literals}
Integer literals represent whole number decimal values using characters \\ \texttt{0 - 9} in a sequence
\begin{lstlisting}
//examples of integer literals:
0
1
354234
\end{lstlisting}

\subsubsection{Float Literals}
Float literals represent whole or non-whole number values. Floats must have a decimal point followed by at least one number \texttt{1 - 9}. Floats may have \texttt{0 - 9} preceding the decimal. Floats may be exponents in which case the sequence is concluded with \texttt{+} or \texttt{-} and \texttt{e} or \texttt{E}  and \texttt{0 - 9}.
\begin{lstlisting}
//examples of float literals:
.3
0.3
3.6
3.6e+3
3.6E-3
\end{lstlisting}

\subsubsection{Boolean Literals}
Boolean literals are \texttt{true} or \texttt{false}. The type \texttt{boolean} has two values that are represented by the boolean literals.

\subsubsection{Character Literals}
\begin{lstlisting}
//examples of character literals:

\end{lstlisting}
\subsubsection{String Literals}
String literals consist of zero or more character sequences enclosed within double quotes, \texttt{"}.  A backslash is used as an escape to represent special characters.
\begin{lstlisting}
//examples of String literals:
"hello world"			// an 11 character string
""                   // an empty string

//special characters:
"\n"  				// a string containing the newline character
"\""					// a string containing the double quote character
"\'"					// a string containing the single quote character
\end{lstlisting}
\subsection{Separators}
Jtrix has the following separators:

\begin{lstlisting}

(  )    {  }    [  ]    ;    ,    .
\end{lstlisting}

\end{document}